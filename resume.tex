\documentclass[]{resume}

\usepackage{xcolor}
\hypersetup{
    colorlinks,
    linkcolor={red!50!black},
    citecolor={blue!50!black},
    urlcolor={black!65}
}

\usepackage{fixltx2e}

%macros defined from here --- https://fortawesome.github.io/Font-Awesome/cheatsheet/

\usepackage{fontawesome}
\newcommand\faGit{{\FA\symbol{"F09B}}}
\newcommand\faUrl{{\FA\symbol{"F0AC}}}
\newcommand\faSkype{{\FA\symbol{"F17E}}}
\newcommand\faXing{{\FA\symbol{"F168}}}
\newcommand\faFemale{{\FA\symbol{"F182}}}
\newcommand\faMale{{\FA\symbol{"F183}}}
\newcommand\faChild{{\FA\symbol{"F1AE}}}
\newcommand\faUni{\FA\symbol{"F19C}}
\newcommand\faGrad{\FA\symbol{"F19D}}
\newcommand\faChain{\FA\symbol{"F0C1}}
\newcommand\faFlask{\FA\symbol{"F0C3}}
\newcommand\faCase{\FA\symbol{"F0B1}}
\newcommand\faData{\FA\symbol{"F1C0}}
\newcommand\faAlpha{\FA\symbol{"F15D}}
\newcommand\faNumeric{\FA\symbol{"F163}}
\newcommand\faPhoto{\FA\symbol{"F03E}}
\newcommand\faSkill{\FA\symbol{"F1E3}}

\newcommand{\ie}{\mbox{i.e.~}}
\newcommand{\eg}{\mbox{e.g.~}}
\newcommand{\cf}{\mbox{c.f.}}
\newcommand{\wrt}{\mbox{w.r.t~}}

\begin{document}

%%%%%%%%%%%%%%%%%%%%%%%%%%%%%%%%%%%%%%
%
%     TITLE NAME
%
%%%%%%%%%%%%%%%%%%%%%%%%%%%%%%%%%%%%%%
\hspace{-1.8cm}
\begin{minipage}{0.5\textwidth}
\begin{flushright}
{\sffamily\fontsize{40pt}{10cm}\selectfont Surajit \normalfont Dutta} 
\end{flushright}
\end{minipage} \hspace{5mm} 
\begin{minipage}{0.5\textwidth}
\vspace{3.5mm}
\begin{flushleft}\vspace{-1mm}
{\hspace{1cm}\textcolor{black!65}{\large\faHome}~Spiegelberg 5, 88090 Immenstaad , Germany} \\[1mm]
{\hspace{1cm}\href{mailto:hello@surajitdutta.com}{\faEnvelopeAlt}~\href{mailto:hello@surajitdutta.com}{hello@surajitdutta.com} \ | \
\color{black!65}{\faPhone}~+49 170 9372197} \\[1.2mm]
\par\color{headings}
{\hspace{1cm}\color{black!65}{\faFlag}~Indian \ | \
\color{black!65}{\faMale}~Male \ | \ 
\color{black!65}{\footnotesize\faChild}~DOB: 09.02.1989}
\end{flushleft}
\end{minipage}

\vspace{5pt}
\noindent\makebox[\linewidth]{\rule{\paperwidth}{0.4pt}}

%\vspace{-2mm}
%%%%%%%%%%%%%%%%%%%%%%%%%%%%%%%%%%%%%%
%
%     COLUMN ONE
%
%%%%%%%%%%%%%%%%%%%%%%%%%%%%%%%%%%%%%%

\begin{minipage}[t]{0.33\textwidth} 

%%%%%%%%%%%%%%%%%%%%%%%%%%%%%%%%%%%%%%
%     EDUCATION
%%%%%%%%%%%%%%%%%%%%%%%%%%%%%%%%%%%%%%

\section{{\faGrad~}Education} 

\subsection{{\faUni~}TU Chemnitz}
\descript{MSc in Automotive Software Engineering}
\location{Sept 2015 | Chemnitz, Germany}
\location{Grade: 2.1}
\endsubsection

\subsection{{\faUni~}Acharya Inst. of Tech.}
\descript{BE in Electronics \& Communication Engineering}
\location{July 2011 | Bangalore, India}
\location{Grade: 1\textsuperscript{st} Class}
\endsubsection

%%%%%%%%%%%%%%%%%%%%%%%%%%%%%%%%%%%%%%
%     LINKS
%%%%%%%%%%%%%%%%%%%%%%%%%%%%%%%%%%%%%%

\section{{\faLink~}Links} 
\LARGE\href{https://de.linkedin.com/in/surajitdutta39/en}{\faLinkedinSign} \ 
\href{https://www.xing.com/profile/Surajit_Dutta2?sc_o=mxb_p}{\faXing} \ 
\href{skype://suraj.roland?call}{\faSkype} \ 
\href{https://github.com/surajroland}{\faGit} \
\href{http://www.surajitdutta.com}{\faUrl}
\endsubsection

%%%%%%%%%%%%%%%%%%%%%%%%%%%%%%%%%%%%%%
%     COURSEWORK
%%%%%%%%%%%%%%%%%%%%%%%%%%%%%%%%%%%%%%

\section{{\faBook~}Coursework}
\subsection{{\faPencil~}Graduate}
\descript{Machine Learning + Lab\\
Artificial Intelligence \\
Image Processing + Lab\\
Embedded Systems \\
Real Time Systems \\
FPGA Lab \\
AUTOSAR Lab (Application Layer)}
\endsubsection

\subsection{{\faPencil~}Undergraduate}
\descript{Digital Circuit + Lab \\ 
Digital Communication + Lab \\ 
Control Systems \\
Radar \& Microwave \\
DSP + Lab \\ 
Microcontroller + Lab \\
Microprocessor + Lab \\ 
Digital Image Processing}
\endsubsection

%%%%%%%%%%%%%%%%%%%%%%%%%%%%%%%%%%%%%%
%     SKILLS
%%%%%%%%%%%%%%%%%%%%%%%%%%%%%%%%%%%%%%

\section{{\faSkill~}Skills}
\subsection{{\faCode~}Programming}
\location{Language}
C \textbullet{} C++ \textbullet{} Matlab \textbullet{} Python \textbullet{} VHDL \\
\location{API, Tool, IDE \& OS}
OpenCV \textbullet{} OpenCL \textbullet{} OpenGL \textbullet{} Matlab/Simulink/Octave \textbullet{} Visual Studio/Eclipse \textbullet{} Linux/Windows \\
\location{Framewok}
Caffe \textbullet{} Torch \textbullet{} Theano \textbullet{} MatConvNet \\
\location{Documentation Framework}
\LaTeX \textbullet{} Doxygen \\
\location{Version Control Client}
Git \textbullet{} TortoiseSVN
\endsubsection

%%%%%%%%%%%%%%%%%%%%%%%%%%%%%%%%%%%%%%
%
%     COLUMN TWO
%
%%%%%%%%%%%%%%%%%%%%%%%%%%%%%%%%%%%%%%

\end{minipage} 
\hfill
\begin{minipage}[t]{0.66\textwidth} 

%%%%%%%%%%%%%%%%%%%%%%%%%%%%%%%%%%%%%%
%     EXPERIENCE
%%%%%%%%%%%%%%%%%%%%%%%%%%%%%%%%%%%%%%

\section{{\faCase~}Experience}

\runsubsection{{\faCode\,\faPhoto\textsuperscript{~\faSearch}~}Object Detection Framework}
\descript{| Industrial Experience }
\location{Apr 2015 – Oct 2015 | CMORE Automotive GmbH, Lindau, Germany}
\vspace{3.5mm}
\begin{tightemize}
\item Pedestrian detection using cascade AdaBoost-SVM
\item Lane marker detection using SVM, Logistic Regression, GMM, Decision Tree
\end{tightemize}
\endsubsection

\runsubsection{{\faPhoto\textsuperscript{~\faSearch}~}Object Detection}
\descript{| Industrial Experience }
\location{May 2014 – Apr 2015 | Continental AG, Lindau, Germany}
%\vspace{\topsep} 
\begin{tightemize}
\item Tricky feature representation of the object to be detected under poor illumination, contrast
\item Improving the machine learning algorithms to model unbiased classifier
\item Performance investigation using Qualitative and Quantitative Evaluations
\end{tightemize}
\endsubsection



\runsubsection{{\faAlpha\faNumeric~}Optical Character Recognition}
\descript{| Vocational Project }
\location{Aug 2010 – Nov 2010 | Acharya Institute of Technology, Bangalore, India}
\begin{tightemize}
\item SIFT-PCA based feature representation of the segmented character images
\item Training muliclass SVM for multiple instances of all different characters
\item Shape-context and $\chi^2$ distance based character matching 
\end{tightemize}
\endsubsection


\runsubsection{{\faCamera~}Face Recognition}
\descript{| Vocational Project }
\location{Nov 2009 – Feb 2010 | Acharya Institute of Technology, Bangalore, India}
\begin{tightemize}
\item Fast detection of faces using fast rejection of negative regions using Haar Wavelets
\item PCA based feature compression and denoising
\item Eigen face based face recognition
\end{tightemize}
\endsubsection

%%%%%%%%%%%%%%%%%%%%%%%%%%%%%%%%%%%%%%
%     RESEARCH
%%%%%%%%%%%%%%%%%%%%%%%%%%%%%%%%%%%%%%

\section{{\faFlask~}Research}
\runsubsection{{\faData~}Machine Learninig on Large Data}
\descript{| Master Thesis}
\location{Nov 2014 - Apr 2015 | Continental AG, Lindau, Germany}
Efficient Learning of Large Imbalanced Training Datasets for Support Vector Machines. This includes developing a vanilla SVM which is based on \\
\begin{tightemize}
\item Stochastic Gradient Descent based  learning
\item Balancing the object classes and subclasses (\eg non-pedestrian, pedestrian-frontal, pedestrian-lateral, pedestrian-truncated) 
\item Optimizing SVM (Runtime + Accuracy)
\end{tightemize}
\endsubsection

\runsubsection{{\faFemale\small\faMale~}Pedestrian Detection}
\descript{| Research Internship}
\location{May 2014 – Oct 2014 | Continental AG, Lindau, Germany}
Pedestrian Detection at Night Scenes using Edge Enhanced Features. An increment of 10\% in average precision has been achieved. The work includes
\begin{tightemize}
\item Concatenation of Random Forest based enhanced edge with HOG features
\item Training SVM model for the pedestrian class
\item Performance evaluation 
\end{tightemize}

\endsubsection

%%%%%%%%%%%%%%%%%%%%%%%%%%%%%%%%%%%%%%
%     References
%%%%%%%%%%%%%%%%%%%%%%%%%%%%%%%%%%%%%%

\section{{\faGroup~}References} 
\begin{minipage}{0.5\textwidth}
\begin{flushleft}
{\large \bf\href{http://patrick-ott.de}{Dr. Patrick Ott}} \\
Dipl-Inf. (FH), HRF (Leeds)\\
Machine Learning Expert\\
\href{http://www.ott-busch.com}{Ott \& Busch Intelligent Solution} \\
\href{mailto:patrick.ott@ott-busch.com}{\small\faEnvelopeAlt}~\href{mailto:patrick.ott@ott-busch.com}{patrick.ott@ott-busch.com}
\end{flushleft}
\end{minipage} \hspace{-1cm}
\begin{minipage}{0.5\textwidth}
\begin{flushleft}
{\large\bf\href{https://www-user.tu-chemnitz.de/~vitay/}{Dr. Julien Vitay}} \\
\href{https://www.tu-chemnitz.de/informatik/KI}{Chair of Artificial Intelligence}\\
\href{https://www.tu-chemnitz.de/informatik}{Faculty of Computer Science}\\
\href{https://www.tu-chemnitz.de}{TU Chemnitz} \\
\href{mailto:julien.vitay@informatik.tu-chemnitz.de}{\small\faEnvelopeAlt}~\href{mailto:julien.vitay@informatik.tu-chemnitz.de}{julien.vitay@informatik.tu-chemnitz.de}
\end{flushleft} 
\end{minipage}
\endsubsection

\end{minipage} 
\end{document}  